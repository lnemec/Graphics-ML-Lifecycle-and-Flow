\documentclass[border=10pt]{standalone}

\usepackage{array}
\usepackage{rotating}
%Farbige gestaltung
\usepackage{xcolor}    %this package provides foreground and background color management

\definecolor{darkblue}{RGB}{0,68,170}
\definecolor{darkred}{RGB}{106,0,0}
\definecolor{darkgreen}{RGB}{1,73,62}
\definecolor{lightgreen}{RGB}{129,195,65}
\definecolor{darkviolett}{RGB}{59,0,112}
\definecolor{violett}{RGB}{136,134,248}
\definecolor{pink}{RGB}{151,8,79}
\definecolor{mediumgreen}{RGB}{0,187,177}
\definecolor{darkgrey}{RGB}{32,32,37}
\definecolor{mediumgrey}{RGB}{152,152,157}
\definecolor{lightgrey}{RGB}{230,230,235}

%-------------------------------------------%

\usepackage{tikz}

\usetikzlibrary{shapes,arrows}
\usetikzlibrary{shapes.geometric}
\usetikzlibrary{shapes.arrows}
\usetikzlibrary{positioning, calc}
\usetikzlibrary{decorations.text}
\usetikzlibrary{backgrounds,fit}

%-------------------------------------------%

\input{../ML-Flow.tikzstyle}
		
\begin{document}
	\pagestyle{empty}

	\begin{tikzpicture}[node distance = 2cm,
										auto,
										background rectangle/.style={fill=white}, 
										show background rectangle]

		%-------------------------------------------%
		\node (infra) [Infra] (infra1){
			\begin{turn}{90}
				Compute \& Data Infrastructure
			\end{turn}
		};
		%-------------------------------------------%
		
		
		%-------------------------------------------%
		\node (infra) [Infra, right of=infra1,xshift=9cm] (infra2){
			\begin{turn}{270}
				git flow; Continuous Integration \& Deployment
			\end{turn}
		};	
		%-------------------------------------------%

		\node [database, above of=infra1, yshift=7.5cm,label=above:Raw Data] (source) {};
		
		\node [outteam, right of=source, xshift=1.5cm] (de) {Data Engineering};
		\node [right of =de](N0){};
		\node [outteam, right of=N0] (tm) {Train Model};

		%-------------------------------------------%
		% empty nodes
		\node [below of =N0, node distance=1cm](N1){};
		\node [left of =N1, node distance=0.3cm](N2){};
		\node [right of =N1, node distance=0.3cm](N3){};

		\path [line,  draw=mediumgrey] (infra2.north) |- node[xshift=-5.5cm]{Infrastructure code}(N1.east);
		\path [arrow,  draw=mediumgrey] (N1.west) -| (infra1.north);
		\draw [line,  draw=mediumgrey] (N2) -- (N3);
		%-------------------------------------------%

		%-------------------------------------------%
		% Start with ML nodes
		\node [below of =N1, node distance=2.0cm](N1a) {};
		\node [io, left of = N1a](access) {Data Access};
		\node [io, right of = N1a](GMLm) {Get ML Model};

		%-------------------------------------------%				
		%		% Place nodes
		\node [block, below of = N1a] (feature) {Feature \\Engineering};
		\node [block, below of = feature, yshift=-0.6cm] (inference) {Inference};
		\node [block, below of = inference, yshift=-0.6cm] (eval) {Evaluating};
		\node [decision, below of=eval, yshift=-0.6cm] (decide) {Good results?};
		\node [cloud, right of=decide] (expert) {Expert};
		\node [database, left of = decide, xshift=-1cm, label=above:Store Results] (store) {};
		\node [io, below of = decide, yshift=-1.0cm] (model) {Retrain Model};

		%-------------------------------------------%	
		%		% Draw edges
		\path [line,dashed, darkgrey] (source) -- (de);
		\path [line,dashed, darkgrey] (de.south) -- (access.north);
		\path [arrow] (access.west) -- node {pipeline} ($(access.west) - (1.6cm,0) $);
		\path [line,dashed, darkgrey] (tm) -- (GMLm);

		%-------------------------------------------%	
		%		% Draw edges
		\path [arrow] (access.south) |- (feature.west);
		\path [arrow] (GMLm.south) |- (inference.east);
		\path [arrow] (feature) --  node(p2){pipeline} (inference);
		\path [arrow] (inference) -- node(p4){pipeline} (eval);
		\path [arrow] (eval) --  node(p5){pipeline} (decide);
		\path [line,dashed, darkgrey] (expert) -- (decide);
		\path [arrow] (decide.west) -- node[xshift=0.2cm] {yes} (store.east);
		\path [arrow] (decide.south) -- node {no} (model.north);
		
		%-------------------------------------------%	
		%		% Draw edges
		\path [line, dashed, draw=mediumgrey] ($(feature.west) - (3.4cm,0) $) -- node {compute} (feature.west);
		\path [line, dashed, draw=mediumgrey] ($(inference.west) - (3.4cm,0) $) -- node {compute} (inference.west);
		\path [line, dashed, draw=mediumgrey] ($(eval.west) - (3.4cm,0) $) -- node {compute} (eval.west);
		%-------------------------------------------%
		\path [line, dashed, draw=mediumgrey] (feature.east) -- node{Source code} ($(feature.east) + (3.4cm,0) $);
		\path [line, dashed, draw=mediumgrey] (inference.east) -- node{Source code} ($(inference.east) + (3.4cm,0) $);
		\path [line, dashed, draw=mediumgrey] (eval.east) -- node{Source code} ($(eval.east) + (3.4cm,0) $);
		%-------------------------------------------%
		
		\node [block,
					minimum width=3.2cm,
					minimum height=0.7cm,
					draw=lightgreen,
					fill=lightgreen!20,
					draw=lightgreen,
					above right=0.1cm and 8cm of access] (monitor1) {Drift Monitor};
		\node [block,
					below left=0.1cm and -1.55cm of monitor1,
					minimum width=1.5cm,
					draw=lightgreen,
					fill=lightgreen!20,
					text width=1.3cm] (monitor2) {Back-end};
		\node [block,
					right=0.1cm of monitor2,
					minimum width=1.5cm,
					draw=lightgreen,
					fill=lightgreen!20,
					text width=1.3cm] (monitor3) {User Interf.};
		\node [block,
					minimum width=3.2cm,
					minimum height=0.7cm,
					draw=mediumgrey, 
					fill=lightgrey!70,
					below left=0.1cm and -1.55cm of monitor3] (monitor4) {Infrastructure};
		\node[ rectangle, 
					rounded corners,
					draw=darkgreen,
					fit=(monitor1) (monitor4)] {};
		
		\node [block,
					minimum width=3.2cm,
					minimum height=0.7cm,
					draw=lightgreen,
					fill=lightgreen!20,
					draw=lightgreen,
					below =1.4cm of monitor4] (monitor5) {Infrastructure Monitor};
		\node [block,
					below left=0.1cm and -1.55cm of monitor5,
					minimum width=1.5cm,
					draw=lightgreen,
					fill=lightgreen!20,
					text width=1.3cm] (monitor6) {Back-end};
		\node [block,
					right=0.1cm of monitor6,
					minimum width=1.5cm,
					draw=lightgreen,
					fill=lightgreen!20,
					text width=1.3cm] (monitor7) {User Interf.};
		\node [block,
					minimum width=3.2cm,
					minimum height=0.7cm,
					draw=mediumgrey, 
					fill=lightgrey!70,
					below left=0.1cm and -1.55cm of monitor7] (monitor8) {Infrastructure};
		\node[ rectangle, 
					rounded corners,
					draw=darkgreen,
					fit=(monitor5) (monitor8)] {};
		
		\node [block,
					minimum width=3.2cm,
					minimum height=0.7cm,
					draw=lightgreen,
					fill=lightgreen!20,
					draw=lightgreen,
					below =1.4cm of monitor8] (monitor9) {Model Monitor};
		\node [block,
					below left=0.1cm and -1.55cm of monitor9,
					minimum width=1.5cm,
					draw=lightgreen,
					fill=lightgreen!20,
					text width=1.3cm] (monitor10) {Back-end};
		\node [block,
					right=0.1cm of monitor10,
					minimum width=1.5cm,
					draw=lightgreen,
					fill=lightgreen!20,
					text width=1.3cm] (monitor11) {User Interf.};
		\node [block,
					minimum width=3.2cm,
					minimum height=0.7cm,
					draw=mediumgrey, 
					fill=lightgrey!70,
					below left=0.1cm and -1.55cm of monitor11] (monitor12) {Infrastructure};
		\node[ rectangle, 
					rounded corners,
					draw=darkgreen,
					fit=(monitor9) (monitor12)] {};
		
		\node [block,
					minimum width=3.2cm,
					minimum height=0.7cm,
					draw=lightgreen,
					fill=lightgreen!20,
					draw=lightgreen,
					below =1.4cm of monitor12] (monitor13) {Automation Monitor};
		\node [block,
					below left=0.1cm and -1.55cm of monitor13,
					minimum width=1.5cm,
					draw=lightgreen,
					fill=lightgreen!20,
					text width=1.3cm] (monitor14) {Back-end};
		\node [block,
					right=0.1cm of monitor14,
					minimum width=1.5cm,
					draw=lightgreen,
					fill=lightgreen!20,
					text width=1.3cm] (monitor15) {User Interf.};
		\node [block,
					minimum width=3.2cm,
					minimum height=0.7cm,
					draw=mediumgrey, 
					fill=lightgrey!70,
					below left=0.1cm and -1.55cm of monitor15] (monitor16) {Infrastructure};
		\node[ rectangle, 
					rounded corners,
					draw=darkgreen,
					fit=(monitor13) (monitor16)] {};

	\end{tikzpicture}
	
\end{document}
